\documentclass[11pt]{iopart}
\usepackage{graphicx}
\usepackage{iopams}

% Fixing iopams poor centering of equations:
\expandafter\let\csname equation*\endcsname\relax
\expandafter\let\csname endequation*\endcsname\relax
\usepackage{amsmath}
%


\begin{document}

\title[]{FYP}

\author{Donovan Webb}

\address{Department of Physics,
University of Bath, Bath BA2 7AY, United Kingdom}
\ead{dw711@bath.ac.uk}

\begin{abstract}
\end{abstract}

\section{Introduction}

\begin{equation}
  y=x
  \end{equation}

\begin{figure}
  \caption{Caption?}
  \end{figure}

%% - Motivation and overview brief description of power transfer problem
\subsection{Maxwell equations}
\subsection{Transformation Optics and Metamaterials}
%% - Wireless power transfer - lit review
%% - Discussion on circuitary, RLC (could leave to methods?)
%% - Where this work fits in overall field

\section{Methods}
\subsection{DC Magnetic Fields}
Helmholtz coils were powered by a constant DC current to create a
uniform magnetic field within their center. A commercially available
XXX Hall probe was zeroed by using a MuMetal cannister, and then
placed at the center of the Helmholtz coils.  A Hall probe relates a
measured Hall voltage, $V_H$, to a surrounding magnetic field, $B$
\cite{XXX} as
\begin{equation}
  V_H = \frac{IB}{net}.
\end{equation}
The probe maintains constant current supply $I$, and material
paramaters $n$ (charge carrier density), $e$ (charge of electron) and
$t$ (thickness of probe) meaning a calibrated probe may give accurate
readings for magnetic fields.\\

The magnetic field, $B$, produced at the center of Helmholtz coils
with radius $R$, seperated by a distance $R$ should follow,
\begin{equation}
  B = \frac{8}{5\sqrt{5}}\frac{\mu_0 nI}{R},
  \label{eqn:helm}
\end{equation}
where $I$ is the current supplied to the coils and $n$ is the number of
turns of wire. This equation follows directly from the Biot-Savart law
\cite{XXX} and the relative geometry of the coils as seen in
figure~\ref{fig:helm}. From equation~\ref{eqn:helm} it can be seen that
the magnetic field should increase linearly with supplied
current. Using the Hall probe we ensured this was the case and found
the relationship of current supplied to magnetic field produced for
our paticular Helmholtz arrangement.\\ Now, with the capability to
produce known external magnetic fields, the described field
concentrating shells may be placed within this field and the Hall
probe may be placed within their inner radius to measure concentrated
field.

\subsection{AC characterization}
Initially the Helmholtz arrangement was repeated for exploration of
the concentrating shells behaviour in alternating magnetic
fields. However, instead of a Hall probe, a small solenoid was used to
detect the oscillating field. From Faraday's law, a voltage will be
induced in a wire loop due to a time dependent magnetic field. A
series of loops constituting a small solenoid will respond to a
sinuisoidal magnetic field, $B = B_0\cos{wt}$, with the relationship,
\begin{equation}
  V = -NAB_0\omega\sin{\omega t},
\end{equation}
where $A$ is the area of one loop and $N$ is the number of loops,
$\omega$ is the angular frequency of the alternating magnetic field
and $t$ is time.\\ As $\omega$ is known and all other parameters
except external field are kept constant, the voltage across the
solenoid may be measured experimentally to find the relative magnetic
field strength.\\
The solenoid must however be characterized in order to find the
absolute magnetic field values. This was done by measurement of the
self inductance, $L$, of the solenoid as, XXX
\begin{equation}
  L = \mu_0\mu_rN^2A/l
\end{equation}
XXX

Due to the induced voltage across the inductor being small and background noise being high, a lock-in amplifier was used to select only the desired signal frequency. This substantially reduced noise in our readings allowing higher frequency and lower magnetic field strength experiments.\\

Use of solenoid, limitations of Helmholtz and pick up.
Use of RLC circuitary.

\subsection{Power Transfer}
Power transfer experiments measure power dropped across a load
resistor in a receiving circuit verses power lost in the transmitting
circuit's inductor. The receiving circuit, seen in
figure~\ref{fig:pt_circuit}, has multiple arrangements to optimise
power transfer. The simplest of which is the load resistor in series
with the receiving inductor. In this case the optimal load resistance
is $R = \omega*L$, where $\omega$ is the angular frequency of the
oscillating magnetic field and $L$ is the circuit inductance.\\
To maximise power transfer an RLC circuit is constructed on the
receiving circuit. Ideally in an RLC circuit the complex impedance of
the inductance and capacitance cancel leaving only the load
resistance. Our inductance is set by the solenoid we chose to use and
so for exploring power transfer at various frequencies, a capacitance
can be found to satisfy the resonance condition. If inductance, $L$, does not
vary then capacitance, $C$, is easily found
by,
\begin{equation}
  C = \frac{1}{w^2L}.
\end{equation}
However, we need not assume that inductance is constant. A series RLC
circuit can be constructed as seen in figure~\ref{fig:RLC_series} to
ensure resonance is met. Resonance occurs in this circuit when $V_A$
is exactly in phase with $V_B$. In this case any imaginary impedances
are cancelled and only real resistance remains. \\ For maximal power
transfer a parallel RLC circuit is preferable however due to the
restraints of the experiement a non-ideal version must be made as seen
in figure~\ref{fig:RLC_parallel}. The resonant condition found by the
series RLC is a close approximation to this more complicated parallel
circuit. \\

To maximise power transfer in the series RLC case, a familiar idea of
impedance matching occurs, i.e. Power is maximised when the load
resistance is equal to any internal resistances of the
components~\cite{XXX}. As internal resistances are difficult to
measure and may depend on current XXX, this was found experimentally
by measuring voltage and current over the load resistance whilst
varying load resistance. \\
For the parallel RLC case

\subsection{COMSOL}
Expleen
%% - Do not gloss over too many details
%% - Do discuss calibration of devices
%% - How to make shell
%% - Helmholtz and Hall probe(skip?)
%% - Circuitary for both DC, AC and power transfer.
%% - RLC parallel tuning

\section{Results}
%% - Key results only!
%% - Discuss observed trends
%%     - Concentration factor
%%     - Power transfer
%% - Discuss sources and magnitude of errors
%% - Simulations
\subsection{DC Magnetic Fields}
\begin{figure}
  \caption{}
  \label{fig:DC_graph}
\end{figure}

Using the DC Helmholtz set up as described in Methods, we observed
constant concentration factors for different shell constructions in an
external magnetic field ranging from $1$ to $22 G$. No shell, 18
MuMetal, 36 MuMetal, 18 Copper and 18 MuMetal + 18 Copper shells were
used and their behaviour with field may be seen in
figure~\ref{fig:DC_graph}.\\ It was found that the shell contruction of 36
MuMetal thin sheets gave the optimum concentration of $C = 2.38$ with
minimal error ($0.1\%$) at higher field strengths and a maximum error
of $4.0\%$ at an external field of $1.4$~G. This increase in error at
low magnetic fields is due to limited sensitivity of our Hall probe
and current measurements over the Helmholtz coils.\\ Similar error
relationships are observed for the other constructions. It should be
noted that we assume the dipole has been placed in the same position
and orientation in all experiments and so errors due to placement are
excluded here.\\ The copper only shell showed no concentration of
internal field as expected. This is due to copper having a relative
permeability similar to air, $\mu_r = 1.0$, and so negligible field
guiding properties. Furthermore, in the DC regime copper will not
shield XXangularXX fields as required by the optimal TO
concentrator.\\


\subsection{AC characterization}
\begin{figure}
  \caption{}
  \label{fig:AC_helm_graph}
\end{figure}

\subsection{Helmholtz}
The Helmholtz coils were supplied with an alternating current in order
to create an alternating magnetic field.  Now using the voltage
induced across solenoid to detect alternating magnetic field strength
as decribed in Methods, the concentration of various shell
arrangements were explored.\\

The concentration factors between $0.5$ and $30$~kHz can be seen in
figure~\ref{fig:AC_helm_graph}. It was found that a mixed shell of
alternating $18$ copper and $18$ MuMetal sheets had the optimum
concentration factor of $C = 3.12$ at $5$~kHz.\\ Here we can see that
the copper sheets now have a modest concentrating effect as frequency
increases. This behaviour is expected as copper will shield
perpendicular alternating magnetic fields which is desired for the
optimal TO concentrator. However, it is suprising that this shielding
occurs at such low frequencies, i.e. Much less than skin depth of
copper.XXX
We found that the copper shell increases in efficacy from $C = 1.0$ at
$50$~Hz until $C = 1.3$ (2SF) at $10$~kHz and does not increase
substantially more as frequency is increased. This suggests that the
shielding effect of the copper depends on frequency but saturates
early.\\ Supporting work done on COMSOL (see Methods) suggests a similar observed
effect where 36 copper sheets increase rapidly in concentration factor
from $C = 1$ at $0$~Hz to $C = 1.5$ (2SF) at $10$~kHz.\\

The strong linear decay of field concentration after $5-10$ kHz for
all but the copper only shell is also a suprising result which,
although could be explained by the MuMetal permeability frequency
response, also appears to occur at too low a frequency.\\ COMSOL work
does not show this relationship and so the source may be either not
modelled appropriately within COMSOL or be a fault in this
experimental design.\\

Apart from instrument and measurement reading errors which constitute
only a small error (XXX\%), we observed errors due to high pick-up in
cables connecting the solenoid to the lock-in amplifier. This source
of noise was at the same frequency as our desired signal and so is
difficult to remove other than careful cable placement and using
shielding. We believe this pick-up was worsened by the fact the
Helmholtz coils must be driven with high voltage and current to
create a useful magnetic field and that the magnetic field was not
localised to just our solenoid and shell but also was subject to the
cabling and any nearby detectors. This prompted a decision to focus on
two dipole coupling experiments as this pick-up error can be greatly
reduced.\\

%To further explore Cu 0--5 kHz retest gave very flat line


\subsection{Power transfer}
First we show the simplest power transfer experiment where the load
resistance is in series with the receiving
inductor. Figure~\ref{fig:RL} shows the relative PTE versus load
resistance for various frequencies. The peaks of these curves confirm
the expected optimum load resistance of $R = \omega
L$. \\
An Oscillating magnetic field, $B$, produced from a solenoid and
concentrated by a shell follows,
$$B = CI\mu_0n\cos{\omega t},$$
where $C$ is the concentration factor, $I$ is the current through the
solenoid and $n$ is the number of turns of the solenoid.  If a second
solenoid is placed within the field of the first, as shown in
figure~\ref{fig:dipole-dipole}, then voltage will be induced across it
according to Faraday's law,
$$V = -NA\frac{dB}{dt},$$
$$V = C\omega I NA\mu_0n\sin{\omega t},$$
where $N$ is the number of turns of the solenoid and $A$ is the area
of one turn.
The power dropped across a resistor with magnitude $\omega L$ in
series with this inductor will then be described by,
$$ P = V^2/R $$
$$ P = \frac{C^2wI^2k}{L}\sin{\omega t}^2$$
where $k$ is the collection of constant coefficients that will remain
constant between different shells.  Max power received in the second
circuit is therefore proportional to $w$, $\frac{C^2}{L}$ and
$I^2$. Plotting $\frac{P}{I^2}$ against angular frequency $w$
therefore gives $\frac{C^2k}{L}$ as shown in XX
figure~\ref{fig:RL_freq}. Assuming the coefficients in $k$ remain
constant, comparisons of this gradient between a concentrating shell
with inductance $L_s$ and no shell with inductance $L_0$ and $C_0 = 1$
yields $\kappa = \frac{C^2L_0}{L_s}$. Figure XXX~\ref{fig:RL_freq}
shows how $\kappa$ depends on frequency for various shell
configurations.\\
As observed in the Helmholtz driven field case, we see that the copper
sheets begin to have a concentrating effect between $0$ and $10$
kHz. The Copper only shell increases to around $\kappa = 2$ which, if
it is assumed that $L_0 = L_s$, corresponds to a power transfer
increase of $2$x or a corresponding field concentration of $\sqrt{2}$
within the shell's cavity. It can be seen that using only MuMetal
sheets gives a power transfer increase of $6$x and a concentration of
field that is independent of field oscillation frequency for the range
$0 - 30$ kHz. This differs from the previous Helmholtz result where a
steady drop off of concentration factor was observed as frequency
increased past $10$ kHz.\\
The mixed shell of 18 MuMetal sheets and 18 Copper sheets was found to
have the best power transfer increase of $9$x after the copper sheet
effectively shields the angular field at $10$ kHz. This power increase
corresponds to a magentic field concentration of $3$ within the shells
cavity.\\

Figure~\ref{fig:RL_freq} XXX shows the optimal PTE for a range of
frequencies with different shell constructions around the receiving
inductor.  Assuming the inductance value remains constant with
different shell configurations (an exploration of this assumption is
considered in DiscussionXXX), absolute power transfer can be
calculated as described in Methods.  Table~\ref{table:RL_freq} gives
the maximal power transfer for different arrangements of shells.\\


\subsection{Other COMSOL}


\section{Discussion}
%% - Tee and Pi circuit analysis
%% - Models
%% - Overall trends matching simulation, analysis and Experimental
%% - Comparison with other published results
%% - Contextualize findings in light of other work

\section{Conclusions}

%ref
\section{References}


\end{document}

