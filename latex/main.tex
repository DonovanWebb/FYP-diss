%SETTINGS{{{
\documentclass[11pt]{iopart}
\usepackage{graphicx}
\usepackage{iopams}

% Fixing iopams poor centering of equations:
\expandafter\let\csname equation*\endcsname\relax
\expandafter\let\csname endequation*\endcsname\relax
\usepackage{amsmath}
%
%}}}
\begin{document}
%TITLE_ABS{{{
\title[]{FYP}

\author{Donovan Webb}

\address{Department of Physics,
University of Bath, Bath BA2 7AY, United Kingdom}
\ead{dw711@bath.ac.uk}

\begin{abstract}
\end{abstract}
%}}}
%INTRO{{{
\section{Introduction}
%% - Motivation and overview brief description of power transfer problem

\subsection{Maxwell equations}
\subsection{Transformation Optics and Metamaterials}
\subsection{Wireless Power Transfer}
%% - Wireless power transfer - lit review
%% - Discussion on circuitary, RLC (could leave to methods?)
%% - Where this work fits in overall field
%%%--------------------------------------------------------------------
%PREV{{{
\begin{abstract}

A device capable of uniformly concentrating a magnetic fields inside of
a free space cavity will increase the efficiency of many magnetic
devices and sensors.  This project shall look at a proposed design for
a magnetic field concentrator informed by the transformation optic
technique. A metamaterial shell comprised of high and low permeability
sections alternating in the angular direction has been shown to
approximate the designed concentrator\cite{N2014}. The ability of the shell acting
as a concentrator will be explored in various regimes with a specific
focus on improving efficiency of wireless power transmission.

\end{abstract}
\section*{Introduction To Physics}
%% Background Physics use about two pages. This is the literature review too!
%% \subsection*{Maxwells equations --- Magnetism}
%% $$\nabla\times \textbf{H} = jw[\epsilon]\textbf{E}$$
%% $$\nabla\times \textbf{E} = -jw[\mu]\textbf{H}$$
\subsection*{Transformation Optics and Metamaterials}
Transformation optics has enabled the design and development of many
new devices such as magnetic cloaks~\cite{Sun2017}, rotators~\cite{Sun2017}, hoses~\cite{Navau2014}, perfect lenses~\cite{Pendry2000} and
magnetic concentrators~\cite{Navau2012}\cite{N2014}. By exploiting the form invariance of Maxwell's
equations, any spatial coordinate transformation may instead be expressed by
the permeability and permittivity of an inserted material~\cite{Pendry2006}. This is shown schematically in figure~\ref{TO}, where $a$ is a ray in free space, $b$ is a ray in a transformed space and $c$ is again in free space but with a specific material inserted. \\
In the static case, this may be described by rearranging,
%% \begin{equation}
%%   \label{ME1}
%%   \hspace{2cm} \nabla'\times \textbf{E'} = -jw[\mu_0]\textbf{H'},
%% \end{equation}
where Faraday's law expressed in a
transformed coordinate space $x'(x, y, z), y'(x,
y, z), z'(x, y, z)$ and $[\mu_0]$ is the permeability of free space,
to
%% \begin{equation}
%%   \label{ME2}
%%   \hspace{2cm} \nabla\times \textbf{E} = -jw[\mu']\textbf{H},
%% \end{equation}
where we are now in Cartesian space $x,y,z$
and $[\mu']$ is a non-free space permeability. A similar method may be used for the other Maxwell equations to find permittivity.\\
\begin{figure}[!t] \centering
  %% \noindent\includegraphics[height=0.25\linewidth]{tran_opt.png}
  \caption{\label{TO} Schematic of the methodology of transformation optics. $a)$ Ray in free Cartesian space. $b)$ Ray in free transformed coordinate system. $c)$ Ray in Cartesian space with material inserted with permeabilities and permittivities found by transformation optics. Figure taken from~\cite{Thesis}.}
  \end{figure}

\noindent This process can be simplified as the required permittivity and permeability may
be found if the Jacobian describing transformation from free space to the transformed space is known. This is described by
%% \begin{equation}
%%   \label{J}
%%   \hspace{2cm} \mu'=\frac{A\mu_0 A^T}{|A|}~~~~~~~~and~~~~~~~~~\epsilon'=\frac{A\epsilon_0 A^T}{|A|}
%% \end{equation}
where $A$ is the Jacobian matrix and $\epsilon$ is relative permittivity. \\

\noindent Resulting permeabilities may be anisotropic, negative and
have arbitrary magnitude when using this method. As bulk materials
often do not have these properties, electromagnetic metamaterials are
often employed.  Metamaterials are materials engineered to have
properties not observed naturally by utilising structures of a scale
much smaller than the wavelength of an interacting field. This results
in an apparent homogeneous material with unusual refractive
indices. Metamaterials with anisotropic and negative~\cite{Pendry2000}
permeabilities and permittivities have been realized.

The focus of this research will be on magnetic concentrators. It is
well known[ref] that ferromagnetic materials concentrate magnetic
fields within their bulk. However this has only limited applications
due to the concentrated region being within a material.
Employing
transformation optics we may solve for an optimised field concentrator
where all the energy of a chosen region $A$ is confined to a
region of free space, $B$. The proposed geometry of such a device is
shown in figure~\ref{conc} $a$.
A cylindrical shell is constructed with inner radius $R_1$ and outer
radius $R_2$ where $A$ is the area between $R_1$ and $R_2$, and $B$ is
the inner region confined by $R_1$. It is assumed that this shell is
infinite along the $z$ direction.  Employing transformation optics we
apply two coordinate transforms: First the region $\rho < R_2 - \xi$
is linearly compressed to region $\rho < R_1$, where $\rho$ is the
radial dimension and $\xi$ is a positive constant.  To ensure the
continuity of our transformed space, a high ($k$th) order polynomial
expansion of a ``skin'' of width $\xi$ is expanded to fill the now
empty region $A$. The limit of $\xi \rightarrow 0$ is taken to
concentrate all of region $B$ into $A$. \\
This process may be described by the following coordinate transforms
where $\theta' = \theta$, $z' = z$, and $\rho'$ is given by
%% \begin{equation}
%%   \label{transform}
%%   \begin{align}
%% \rho' = \frac{R_1}{R_2-\xi}\rho,~~~~~~~~\rho'\in[0,~R_2-\xi)~~\\
%% \rho' = R_2^{1-k}\rho^k.~~~~~~~~~\rho'\in[R_2-\xi,~R_2)
%%   \end{align}
%% \end{equation}
Finding the Jacobians of these transforms and utilising
equation~\ref{J} we find that the following matrices describe the
required relative permeabilities of an inserted material,
%% \begin{equation}
%%   \label{mat}
%%   \begin{split}
%%  \mu' = \begin{pmatrix}1&0&0\\0&1&0\\0&0&(\frac{R_2-\xi}{R_1})^2\end{pmatrix}~~~~~~~~~~~~~\rho'\in[0,~R_1)~~\\
%% ~~~~~~\mu' = \begin{pmatrix}k&0&0\\0&1/k&0\\0&0&\frac{1}{k}(\frac{\rho'}{R_2})^{2/k-2}\end{pmatrix}~~~~~\rho'\in[R_1,~R_2)
%%   \end{split}
%% \end{equation}
By matching boundary conditions at $R_2-\xi$ and taking the limit
$\xi\rightarrow 0$ we find $k \rightarrow \infty$. The $z$ reliance on
$\mu'$ may be ignored as we assume the infinite tube to be invariant
in $z$.  This results in the permeability within $B$ being satisfied
by free space and a material with $\mu_{\rho} \rightarrow \infty,~
\mu_{\theta} \rightarrow 0$ satisfies the requirements for region $A$.\\
This is where transformation optics ends and the exploration of
metamaterials satisfying these equations begins. It was proposed~\cite{Navau2012}
that the guiding of field lines by ferromagnetic material ($\mu >> 1$)
and exclusion of field lines by SC material ($\mu \rightarrow 0$) can
be arranged as shown in figure~\ref{conc} $b$ to approximate such a material with
high radial and near zero angular permeability.\\ In a uniform
external field the theoretical shell should increase field within
region $B$ by a factor of $R_2/R_1$, however the approximated shell,
shown in figure~\ref{conc}, was shown experimentally to only yield a
concentration factor of $2.7$ ($R_2/R_1=4$). This suggests that the
approximation is far from optimised. \\ If an internal magnetic field
is considered using a dipole, all flux within $A$ is concentrated to
the outside of the shell. The magnetic moment of the dipole appears to
be increased by a factor of $R_2/R_1$ to an observer at $\rho>R_2$.
\subsection*{Magnetic Concentrators}
\begin{figure}[!tb] \centering
  %% \noindent\includegraphics[width=0.75\linewidth]{shells.pdf}
  \caption{\label{conc} $a)$ The geometry of the described problem. All of magnetic energy in region $A$ will be concentrated into region $B$. $b)$ An approximation to the required permeabilities described in equation~\ref{mat}. MuMetal provides large radial permeability whilst superconducting sheets (SC) restrict angular permeability.}
\end{figure}

\subsection*{Coupling --- Wireless charging}
Wireless power transmission is needed for powering devices which are
inconvenient or dangerous to power with wires. Examples include mobile
phones and implanted medical devices. \\
It was suggested~\ref{} that two of the above described shells may be utilised for power
transfer due to the ability to increase coupling of two solenoids
within the inner radii of the shells. One solenoid is supplied with an
alternating current to create an alternating magnetic field. This
field is then concentrated by the second shell and a current is
induced in the second solenoid.\\
Two assumptions must be made: first, the metamaterials
act appropriately at the required frequency in an AC field and second, that the transformation optic
approach is still relevant in non-static fields.\\

\noindent In the case of wireless power transfer there may be a set distance, $d$,
where no material may exist. Shells take up physical space and so it
is worth considering if such shells still offer greater coupling in
this regime compared to bringing the dipoles $d$ apart. As the shell increases the dipoles apparent magnetic moment by $R_2/R_1$, the shell at a distance $d+R_2$ is shown to provide an advantage over bare
dipoles when $\frac{R_2}{R_1} > 1+\frac{R_2}{d}$. For a given $R_2$ this means that $R_1$ may always be reduced to satisfy this inequality.
%%%--------------------------------------------------------------------
%}}}

%}}}
%METHODS{{{ 
\section{Methods}
\subsection{DC Magnetic Fields}
Helmholtz coils were powered by a constant DC current to create a
uniform magnetic field within their center. A commercially available
XXX Hall probe was zeroed by using a MuMetal cannister, and then
placed at the center of the Helmholtz coils.  A Hall probe relates a
measured Hall voltage, $V_H$, to a surrounding magnetic field, $B$
\cite{XXX} as
\begin{equation}
  V_H = \frac{IB}{net}.
\end{equation}
The probe maintains constant current supply $I$, and material
paramaters $n$ (charge carrier density), $e$ (charge of electron) and
$t$ (thickness of probe) meaning a calibrated probe may give accurate
readings for magnetic fields.\\

The magnetic field, $B$, produced at the center of Helmholtz coils
with radius $R$, seperated by a distance $R$ should follow,
\begin{equation}
  B = \frac{8}{5\sqrt{5}}\frac{\mu_0 nI}{R},
  \label{eqn:helm}
\end{equation}
where $I$ is the current supplied to the coils and $n$ is the number of
turns of wire. This equation follows directly from the Biot-Savart law
\cite{XXX} and the relative geometry of the coils as seen in
figure~\ref{fig:helm}. From equation~\ref{eqn:helm} it can be seen that
the magnetic field should increase linearly with supplied
current. Using the Hall probe we ensured this was the case and found
the relationship of current supplied to magnetic field produced for
our paticular Helmholtz arrangement.\\ Now, with the capability to
produce known external magnetic fields, the described field
concentrating shells may be placed within this field and the Hall
probe may be placed within their inner radius to measure concentrated
field.

\subsection{AC characterization}
Initially the Helmholtz arrangement was repeated for exploration of
the concentrating shells behaviour in alternating magnetic
fields. However, instead of a Hall probe, a small solenoid was used to
detect the oscillating field. From Faraday's law, a voltage will be
induced in a wire loop due to a time dependent magnetic field. A
series of loops constituting a small solenoid will respond to a
sinuisoidal magnetic field, $B = B_0\cos{wt}$, with the relationship,
\begin{equation}
  V = -NAB_0\omega\sin{\omega t},
\end{equation}
where $A$ is the area of one loop and $N$ is the number of loops,
$\omega$ is the angular frequency of the alternating magnetic field
and $t$ is time.\\ As $\omega$ is known and all other parameters
except external field are kept constant, the voltage across the
solenoid may be measured experimentally to find the relative magnetic
field strength.\\
The solenoid must however be characterized in order to find the
absolute magnetic field values. This was done by measurement of the
self inductance, $L$, of the solenoid as, XXX
\begin{equation}
  L = \mu_0\mu_rN^2A/l
\end{equation}
XXX

Due to the induced voltage across the inductor being small and background noise being high, a lock-in amplifier was used to select only the desired signal frequency. This substantially reduced noise in our readings allowing higher frequency and lower magnetic field strength experiments.\\

Use of solenoid, limitations of Helmholtz and pick up.
Use of RLC circuitary.

\subsection{Power Transfer}
Power transfer experiments measure power dropped across a load
resistor in a receiving circuit verses power lost in the transmitting
circuit's inductor. The receiving circuit, seen in
figure~\ref{fig:pt_circuit}, has multiple arrangements to optimise
power transfer. The simplest of which is the load resistor in series
with the receiving inductor. In this case the optimal load resistance
is $R = \omega*L$, where $\omega$ is the angular frequency of the
oscillating magnetic field and $L$ is the circuit inductance.\\
To maximise power transfer an RLC circuit is constructed on the
receiving circuit. Ideally in an RLC circuit the complex impedance of
the inductance and capacitance cancel leaving only the load
resistance. Our inductance is set by the solenoid we chose to use and
so for exploring power transfer at various frequencies, a capacitance
can be found to satisfy the resonance condition. If inductance, $L$, does not
vary then capacitance, $C$, is easily found
by,
\begin{equation}
  C = \frac{1}{w^2L}.
\end{equation}
However, we need not assume that inductance is constant. A series RLC
circuit can be constructed as seen in figure~\ref{fig:RLC_series} to
ensure resonance is met. Resonance occurs in this circuit when $V_A$
is exactly in phase with $V_B$. In this case any imaginary impedances
are cancelled and only real resistance remains. \\ For maximal power
transfer a parallel RLC circuit is preferable however due to the
restraints of the experiement a non-ideal version must be made as seen
in figure~\ref{fig:RLC_parallel}. The resonant condition found by the
series RLC is a close approximation to this more complicated parallel
circuit. \\

To maximise power transfer in the series RLC case, a familiar idea of
impedance matching occurs, i.e. Power is maximised when the load
resistance is equal to any internal resistances of the
components~\cite{XXX}. As internal resistances are difficult to
measure and may depend on current XXX, this could also be found experimentally
by measuring voltage and current over the load resistance whilst
varying load resistance. \\
For the parallel RLC case, a more complicated expression for optimal
load resistance was found which depends non trivially on a combination
of internal resistances. A model is proposed below, however,
experimentally locating the optimal resistances was chosen as
measuring internal resistances proved difficult and time consuming.\\

\subsection{COMSOL}
Expleen
%% - Do not gloss over too many details
%% - Do discuss calibration of devices
%% - How to make shell
%% - Helmholtz and Hall probe(skip?)
%% - Circuitary for both DC, AC and power transfer.
%% - RLC parallel tuning
%}}}
%RESULTS{{{
\section{Results}
%% - Key results only!
%% - Discuss observed trends
%%     - Concentration factor
%%     - Power transfer
%% - Discuss sources and magnitude of errors
%% - Simulations
\subsection{DC Magnetic Fields}
\begin{figure}
  \caption{}
  \label{fig:DC_graph}
\end{figure}

Using the DC Helmholtz set up as described in Methods, we observed
constant concentration factors for different shell constructions in an
external magnetic field ranging from $1$ to $22 G$. No shell, 18
MuMetal, 36 MuMetal, 18 Copper and 18 MuMetal + 18 Copper shells were
used and their behaviour with field may be seen in
figure~\ref{fig:DC_graph}.\\ It was found that the shell contruction of 36
MuMetal thin sheets gave the optimum concentration of $C = 2.38$ with
minimal error ($0.1\%$) at higher field strengths and a maximum error
of $4.0\%$ at an external field of $1.4$~G. This increase in error at
low magnetic fields is due to limited sensitivity of our Hall probe
and current measurements over the Helmholtz coils.\\ Similar error
relationships are observed for the other constructions. It should be
noted that we assume the dipole has been placed in the same position
and orientation in all experiments and so errors due to placement are
excluded here.\\ The copper only shell showed no concentration of
internal field as expected. This is due to copper having a relative
permeability similar to air, $\mu_r = 1.0$, and so negligible field
guiding properties. Furthermore, in the DC regime copper will not
shield XXangularXX fields as required by the optimal TO
concentrator.\\


\subsection{AC characterization}
\begin{figure}
  \caption{}
  \label{fig:AC_helm_graph}
\end{figure}

\subsection{Helmholtz}
The Helmholtz coils were supplied with an alternating current in order
to create an alternating magnetic field.  Now using the voltage
induced across solenoid to detect alternating magnetic field strength
as decribed in Methods, the concentration of various shell
arrangements were explored.\\

The concentration factors between $0.5$ and $30$~kHz can be seen in
figure~\ref{fig:AC_helm_graph}. It was found that a mixed shell of
alternating $18$ copper and $18$ MuMetal sheets had the optimum
concentration factor of $C = 3.12$ at $5$~kHz.\\ Here we can see that
the copper sheets now have a modest concentrating effect as frequency
increases. This behaviour is expected as copper will shield
perpendicular alternating magnetic fields which is desired for the
optimal TO concentrator. However, it is suprising that this shielding
occurs at such low frequencies, i.e. Much less than skin depth of
copper.XXX
We found that the copper shell increases in efficacy from $C = 1.0$ at
$50$~Hz until $C = 1.3$ (2SF) at $10$~kHz and does not increase
substantially more as frequency is increased. This suggests that the
shielding effect of the copper depends on frequency but saturates
early.\\ Supporting work done on COMSOL (see Methods) suggests a similar observed
effect where 36 copper sheets increase rapidly in concentration factor
from $C = 1$ at $0$~Hz to $C = 1.5$ (2SF) at $10$~kHz.\\

The strong linear decay of field concentration after $5-10$ kHz for
all but the copper only shell is also a suprising result which,
although could be explained by the MuMetal permeability frequency
response, also appears to occur at too low a frequency.\\ COMSOL work
does not show this relationship and so the source may be either not
modelled appropriately within COMSOL or be a fault in this
experimental design.\\

Apart from instrument and measurement reading errors which constitute
only a small error (XXX\%), we observed errors due to high pick-up in
cables connecting the solenoid to the lock-in amplifier. This source
of noise was at the same frequency as our desired signal and so is
difficult to remove other than careful cable placement and using
shielding. We believe this pick-up was worsened by the fact the
Helmholtz coils must be driven with high voltage and current to
create a useful magnetic field and that the magnetic field was not
localised to just our solenoid and shell but also was subject to the
cabling and any nearby detectors. This prompted a decision to focus on
two dipole coupling experiments as this pick-up error can be greatly
reduced.\\

%To further explore Cu 0--5 kHz retest gave very flat line


\subsection{Power transfer}
First we show the simplest power transfer experiment where the load
resistance is in series with the receiving
inductor. Figure~\ref{fig:RL} shows the relative PTE versus load
resistance for various frequencies. The peaks of these curves confirm
the expected optimum load resistance of $R = \omega
L$. \\
An Oscillating magnetic field, $B$, produced from a solenoid and
concentrated by a shell follows,
$$B = CI\mu_0n\cos{\omega t},$$
where $C$ is the concentration factor, $I$ is the current through the
solenoid and $n$ is the number of turns of the solenoid.  If a second
solenoid is placed within the field of the first, as shown in
figure~\ref{fig:dipole-dipole}, then voltage will be induced across it
according to Faraday's law,
$$V = -NA\frac{dB}{dt},$$
$$V = C\omega I NA\mu_0n\sin{\omega t},$$
where $N$ is the number of turns of the solenoid and $A$ is the area
of one turn.
The power dropped across a resistor with magnitude $\omega L$ in
series with this inductor will then be described by,
$$ P = V^2/R $$
$$ P = \frac{C^2wI^2k}{L}\sin{\omega t}^2$$
where $k$ is the collection of constant coefficients that will remain
constant between different shells.  Max power received in the second
circuit is therefore proportional to $w$, $\frac{C^2}{L}$ and
$I^2$. Plotting $\frac{P}{I^2}$ against angular frequency $w$
therefore gives $\frac{C^2k}{L}$ as shown in XX
figure~\ref{fig:RL_freq}. Assuming the coefficients in $k$ remain
constant, comparisons of this gradient between a concentrating shell
with inductance $L_s$ and no shell with inductance $L_0$ and $C_0 = 1$
yields $\kappa = \frac{C^2L_0}{L_s}$. Figure XXX~\ref{fig:RL_freq}
shows how $\kappa$ depends on frequency for various shell
configurations.\\
As observed in the Helmholtz driven field case, we see that the copper
sheets begin to have a concentrating effect between $0$ and $10$
kHz. The Copper only shell increases to around $\kappa = 2$ which, if
it is assumed that $L_0 = L_s$, corresponds to a power transfer
increase of $2$x or a corresponding field concentration of $\sqrt{2}$
within the shell's cavity. It can be seen that using only MuMetal
sheets gives a power transfer increase of $6$x and a concentration of
field that is independent of field oscillation frequency for the range
$0 - 30$ kHz. This differs from the previous Helmholtz result where a
steady drop off of concentration factor was observed as frequency
increased past $10$ kHz.\\
The mixed shell of 18 MuMetal sheets and 18 Copper sheets was found to
have the best power transfer increase of $9$x after the copper sheet
effectively shields the angular field at $10$ kHz. This power increase
corresponds to a magentic field concentration of $3$ within the shells
cavity.\\

Figure~\ref{fig:RL_freq} XXX shows the optimal PTE for a range of
frequencies with different shell constructions around the receiving
inductor.  Assuming the inductance value remains constant with
different shell configurations (an exploration of this assumption is
considered in DiscussionXXX), absolute power transfer can be
calculated as described in Methods.  Table~\ref{table:RL_freq} gives
the maximal power transfer for different arrangements of shells.\\

Parallel RLC circuits are more fitting for optimising power
transfer~\cite{XXX}. For the arrangement described in
figure~\ref{fig:Parallel_RLC}, a shell comprised of $18$ MuMetal and $18$
Copper sheets was explored. The optimal load resistance was found by
taking voltage measurements across a range of load resistance. An
example power versus load resistance curve for $30$ kHz can be seen in
XXX figure~\ref{fig:parallel_R}.\\
Optimal load resistances were found for a range of frequencies and PTE
were calculated as shown in XXX figure~\ref{fig:parallel_R}. Figure
XX~\ref{fig:parallel_R} shows the ratio of shell present versus no
shell present for the range of frequenies. It can be seen that the
increase of ratio between $0$ and $10$ Hz is still present, however
due to the high error and few data points, other trends are hard to
distinguish. In this arrangement, with the coils seperated by a
distance of XXX mm, maximum observed power transfer is XX $0.05\%$.\\

To further explore PTE, the distance between the two coils was
varied. With a distance of XXX mm and a full shell around the
receiving coil, a PTE of XXX\% was achieved.\\

It was expected that a shell around the transmitting coil would
further increase the field incident on the receiving coil. Therefore
the arrangement described in Methods Figure~\ref{fig:two_shell} was
constructed and the peak power transfer observed at $30510$ kHz was
found to be $1.01\%$.\\

% INCLUDE? A summary of PTE and field concentration factors is shown in the table~\fig{table:summary}.\\


\subsection{Other COMSOL}

%}}}
%DISCUSSION{{{
\section{Discussion}
%% - Tee and Pi circuit analysis
%% - Models
%% - Overall trends matching simulation, analysis and Experimental
%% - Comparison with other published results
%% - Contextualize findings in light of other work
%}}}
%CONCLUSIONS{{{
\section{Conclusions}
%}}}
%REF{{{
\section{References}
%}}}
\end{document}

